\documentclass[10pt]{article}
\usepackage[usenames]{color} %used for font color
\usepackage{amssymb} %maths
\usepackage{amsmath} %maths
\usepackage[utf8]{inputenc} %useful to type directly diacritic characters
\begin{document}
\title{\vskip-0.6in The Steel Story}
\author{Taylor Dunne}
\date{\today}

\large\textbf{Timeline}\\
\normalsize\textbf{1960s}\\
\normalsize 1964 - 1969: steel imports increased from 7.3 percent to 16.7 percent\\
1969: steel imports disrupt the tight grip US steelmakers have on domestic market and producers began to seek protection but out of fear of a trade war no real policy could be made - negotiation of VRAs wit the EC and Japan (scheduled to last until 1974)\\
Late 1960s Early 1970s: restructuring in the industry with major companies in the industry combining (\textbf{Wheeling-Pittsburgh Corporation and National and Granite City Steel})\\
\normalsize\textbf{1970s}\\
1971: key actor - \textbf{USWA president I.W. Abel and deputy under secretary of state Nathaniel Samuels} fighting for an extension of VRA but Consumers Union fought for liberalization and cheaper prices arguing for avoiding inflation\\
1971 and 1973: devaluation of the dollar took the wind out of the protectionist coalition\\  
1972: elections for congress and president - top steel executives funneled more than a quarter of a million dollars into campaigns and slightly less into campaigns for \textbf{Nixon and George McGovern}\\
1973: \textbf{Stewart S. Cort} makes a testimony before the House Committee on Ways and Means arguing 3 main factors that were in favor of protecting steel: (1) exchange rates making foreign steel expensive (2) Experimental Negotiating Agreement (3) temporary increase in world demand. 4 demands were made: (1) president’s decisions to raise or lower tariffs should take into account those contributing to the trade deficit that could be harmed by increased imports (such as steel) (2) and (3) mandatory presidential action against imports after the Tariff commission decided imported steel contributed to injury (rather than waiting for proof that it was the cause) (4) reduce the tariff commissions discretionary powers by establishing specific criteria mandating the commission's action\\
1974: representative James Burke warned against giving the president more power but Trade Act of 1974 shifted power away from congress and toward the president through this the president gained power to renegotiate GATT which was good for those who want free trade but the protectionist gained concessions including amendments to the anti-dumping act which introduced the cost of production tests; president is required to conduct product by product negotiations and set up special trade advisory committees\\ 
1974 - 1975: oil crisis leads to a slow down in the industry growth but also makes the political economic climate more hospitable to protectionism\\  
1977 - Inauguration of Trigger Price Mechanism for all steel imports
Late 1970s: Congressmen from steel producing areas create the Congressional Steel Caucus to press for strict import quotas\\
\normalsize\textbf{1980s}\\
1982:\\ 
January - dozens of antidumping and countervailing duty petitions filed against EC countries\\
October - Negotiations with the EC (scheduled to last through Dec 1985)\\
1984:\\
Severe drop in steel sector employees since 1960 making steel industry concerns far less important for presidential campaigns 6\\
January - Escape clause petition filed by Bethlehem Steel and United Steelworkers\\ 
July - ITC rules affirmatively in the escape clause petition in 5 out of 9 product categories\\
September - Negotiations of VRAs on all nine steel products in escape clause petition: market share for participating nations 18.4 percent \\
1988 - \textbf{Candidate Bush} promises to continue VRA\\
1989 - Steel liberalization program - Coalition of American Steel Using Manufacturers make the argument that there are more people employed by Steel Using Manufacturers than steel producers (headed by Caterpillar inc) 
\normalsize\textbf{1990s}\\ 
1992:\\
April - termination of VRA; breakdown of MSA over allowable subsidies\\
June - Antidumping and countervailing petitions filed against flat rolled products\\
1993:\\
July - ITC rules affirmatively on only a subset of steel industry petitions\\
\normalsize\textbf{FUTURE}\\ 
The evolution of mini mills and other technologies are weakening the political force of the steel industry\\
\large\textbf{Book Notes}\\
\normalsize\underline{The Political Economy of Trade Protection}\\
\normalsize By: Anne O. Krueger\\
Chapter 2: The Rise and Fall of Big Steel’s Influence on U.S. Trade Policy
By: Michael O. Moore 
\begin{itemize}
  \item Protectionist steel import regimes: 1969, 1974, 1977, 1982, 1984\\
  \item “The main source of this political strength was the cohesive coalition of vertically-integrated carbon-steel producers, the steelworkers’ union, and member of congress from steel-producing regions” (15)\\
  \item “Another factor that contributed to steel industry political effectiveness was the relative lack of cohesiveness among domestic interests opposing steel protection in particular steel-using manufacturing industries” (15)\\
  \item The power of the U.S. integrated steel structure begins to go away:\\ 
        1989 - industry forced to accept a much less restrictive VRA\\
        1993 - instead of lobbying for a special import regime the industry relied on administrative protection\\
  \item Changes in factors that earlier led to sector’s political cohesiveness:\\
        Rapidly evolving market structure in the US; “Minimills”; Resulting geographic dispersion\\
        Drop in the number of steelworkers (thus less voters)\\
        Increase in efficiency leads to less of a need for special protection\\
        Opposition (steel using producers) have become more organized\\
Steel sector is one of the most important in the import protection\\

\normalsize\underline{The Political Economy of American Trade Policy}\\
\normalsize Actors:\\
\begin{itemize}
  \item U.S. Steel, LTV, Bethlehem, United Steelworkers of America\\
  \item American Iron and Steel Institute\\
  \item Steel - triangle: steelworkers, integrated steel firms, steel community congressional representatives (Important factors in their success (1) relatively small number of actors (2) immobility of employed factors - low transaction costs)\\
\end{itemize}\\
\normalsize Attempts:\\
\begin{itemize}
  \item Pressure on congress for direct legislative relief\\ 
  \item Lobbying for multilateral steel agreements\\ 
  \item Hundreds of anti-dumping petitions\\ 
  \item Assistance to the integrated sector:\\
  \item Relaxed pollution requirement, Raising price of electricity to hurt the minimills, Reducing labor costs\\
  \item Import restriction\\ 
        Minimills can freeride\\
\end{itemize}
\normalsize\underline{Trade Talks: Episode 24 "The Trump Administration Views Trade as a National Security Threat"}\\
\normalsize February 21, 2018
\begin{itemize}
  \item Trump wants to cut steel imports by 12 percent of America?s demand 
  \item Commerce department laid out 3 options:
\begin{itemize}
\item Quotas - more binding / restraining
\item Tariffs - largely more substantial price effect than quantity effect; easier to allocate the revenue 
\begin{itemize}
  \item Applied on a non-discriminatory basis attempting to get steel production to a certain level in the US
\end{itemize}
\item Target big threats and hit them with major tariffs (53 percent) - 53 percent tariff on 12 countries: Brazil, Korea, Russia, Turkey, India, Vietnam, China, Thailand, Egypt, Costa Rica, S. Africa etc. - not the top exporters to the USA - violation of WTO rules by selecting certain countries ?exemption for national security? Article 21 of GATT but WTO has to decide if they accept that the US is doing it for national security reasons but this opens the door for other countries to do the same thing; If the WTO denies then the US will react negatively; Everyone else will still be affected negatively  
\end{itemize}  
  \item Blanket tariff of 24 percent is big for steel; quota across the board is cutting out 37 percent off of what came in last year also big
  \item Decisions in the past to invoke this law was to avoid dependence on non trustworthy country
  \begin{itemize}
    \item Steel is only 20 percent imports by volume used; Imports from Canada and trustworthy countries
  \end{itemize}
  \item International trade commission Safeguards investigation of steel in 2001 
\begin{itemize}
  \item Made clear what was the scope of investigation; Gathered a lot of facts (questionnaires); Conduct significant hearing (9 days long); Trades office ran exclusions processes
\end{itemize}
  \item This investigation had no questionnaires and there was a 2 hour long hearing - heavy reliance on economic modeling
  \item Scope of products: usually trade remedies are precise, in this case all that was said was ?steel? 
\begin{itemize}
  \item Now we know it is very broad scope; Very unclear throughout the reports what areas of steel are being focused on
\end{itemize}
  \item Needs are very different for varying types of steel
\begin{itemize}
  \item It is possible that a blanket policy on steel can hurt the industry more than help it by not dividing out certain parts
\end{itemize} 
  \item Goal: raise volume produced and prices, but this may not occur in places where they need 
  \item Companies within the US steel industry 
\begin{itemize}
  \item Blast oxygen furnace method
  \item Integrated electric arc minimill: Uses scrap steel and putting into a large vat and recast the steel with electricity; They can now make an array of products 
\end{itemize} 
  \item Big change in employment - how much steel is produced per man hour is much higher
  \item Losers: those that consume steel - autos, farms, etc. (downstream industries), could affect infrastructure projects such as building bridges
  \item Countries who feel targeted will retaliate and others will suffer (not just steel consumers)
  \item Something has to happen because of overcapacity
\end{itemize}

\large\textbf{Other Book Suggestions}\\
\underline{Politics Pressures and the Tariff}\\
By: E.E. Schattschneider 1963\\
\underline{The Decline of American Steel}\\
By: Paul A. Tiffany\\
\underline{Running Steel, Running America}\\
By: Judith Stein\\
\underline{Big Steel and the Wilson Administration}\\
By: Melvin I. Urofsky\\

\end{document}